\documentclass{article}
\usepackage{fancyhdr}
\usepackage{extramarks}
\usepackage{amsmath}
\usepackage{amsthm}
\usepackage{amsfonts}
\usepackage{array}
\usepackage{tikz}
\usepackage[plain]{algorithm}
\usepackage{algpseudocode}
\usepackage{enumitem}
\usepackage{pgfplots}
\usepackage{steinmetz}
\usepackage{siunitx}
\usepackage{subcaption}
\pgfplotsset{}
\usepackage{tabularx}

\topmargin=-0.45in
\evensidemargin=0in
\oddsidemargin=0in
\textwidth=6.5in
\textheight=9.0in
\headsep=0.25in

\linespread{1.1}

\pagestyle{fancy}
\renewcommand\headrulewidth{0.4pt}
\renewcommand\footrulewidth{0.4pt}

\setlength\parindent{0pt}
\newcolumntype{C}[1]{>{\centering\arraybackslash}p{#1}}
\newcolumntype{R}[1]{>{\raggedleft\arraybackslash}p{#1}}
\setlength{\tabcolsep}{0mm}

\lhead{Group P}
\chead{Computer Organization Project}
\rhead{ISA Reference List}


\begin{document}
\section{Instruction Set Reference List}
\bigskip

\begin{minipage}[t]{0.3\textwidth}
	\begin{center}
		\medskip
		\textbf{Formats}
	\end{center}
\end{minipage}
\begin{minipage}[t]{110mm}
	\begin{tabular}{p{12mm}p{15mm}p{10mm}p{10mm}p{10mm}p{22.5mm}R{22.5mm}}
		& 31 & 25 & 21 & 17 & 13 & 0 \\ 
	\end{tabular} 

	\begin{tabular}{p{12mm}|C{15mm}|C{10mm}|C{10mm}|C{10mm}|C{45mm}|}
		\cline{2-6}
		c-type: & opcode & rd & rs1 & rs2 & 0\\ 
		\cline{2-6}
	\end{tabular} 
	
	\medskip

	\begin{tabular}{p{12mm}|C{15mm}|C{10mm}|C{10mm}|C{55.15mm}|}
		\cline{2-5}
		i-type: & opcode & rd & rs & 18-bit immediate \\
		\cline{2-5}
	\end{tabular}

	\medskip
	
	\begin{tabular}{p{12mm}|C{15mm}|C{10mm}|C{65.3mm}|}
		\cline{2-4}
		l-type: & opcode & rd & 22-bit immediate \\
		\cline{2-4}
	\end{tabular}
	

\end{minipage}

\bigskip\bigskip\bigskip

\begin{minipage}[t]{0.3\textwidth}
	\begin{center}
		\textbf{Instructions}
	\end{center}
\end{minipage}
\begin{minipage}[t]{110mm}
	\begin{tabular}{p{12mm}|C{15mm}|C{75.15mm}|}
		\cline{2-3}
		NOP & 000000 & 0 \\ 
		\cline{2-3}
	\end{tabular} 
	
	\medskip

	\begin{tabular}{p{12mm}|C{15mm}|C{10mm}|C{10mm}|C{55.15mm}|}
		\cline{2-5}
		SET & 000001 & rd & rs & 18-bit immediate \\
		\cline{2-5}
	\end{tabular}

	\medskip
	
	\begin{tabular}{p{12mm}|C{15mm}|C{10mm}|C{65.3mm}|}
		\cline{2-4}
		LOAD & 000010 & rd & 22-bit immediate \\
		\cline{2-4}
	\end{tabular}

	\medskip

	\begin{tabular}{p{12mm}|C{15mm}|C{10mm}|C{10mm}|C{55.15mm}|}
		\cline{2-5}
		MOV & 000011 & rd & rs & 0  \\
		\cline{2-5}
	\end{tabular}

	\medskip

	\begin{tabular}{p{12mm}|C{15mm}|C{10mm}|C{10mm}|C{10mm}|C{45mm}|}
		\cline{2-6}
		FADD & 000011 & rd & rs1 & rs2 & 0 \\
		\cline{2-6}
	\end{tabular}

	\medskip

	\begin{tabular}{p{12mm}|C{15mm}|C{10mm}|C{10mm}|C{10mm}|C{45mm}|}
		\cline{2-6}
		FSUB & 000100 & rd & rs1 & rs2 & 0 \\
		\cline{2-6}
	\end{tabular}

	\medskip

	\begin{tabular}{p{12mm}|C{15mm}|C{10mm}|C{10mm}|C{10mm}|C{45mm}|}
		\cline{2-6}
		NEG & 000101 & rd & rs1 & 0 & 0 \\
		\cline{2-6}
	\end{tabular}


\end{minipage}

\pagebreak

\huge\textbf{FADD}\\
\noindent\rule{10cm}{0.4pt}
\normalsize

Format c-type: ADD, rd, rs, rs\\
Format i-type: ADD, rd, rs, immediate\\
Purpose: To add two 32 bit integers
\end{document}
